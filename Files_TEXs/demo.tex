\documentclass[a4paper,11pt]{jsarticle}

% === パッケージの召喚 ===
\usepackage[dvipdfmx]{graphicx} % 画像埋め込み
\usepackage{amsmath, amssymb}   % 数式・記号
\usepackage{url}                % URL表示
\usepackage{bm}                 % ベクトル太字
\usepackage{ascmac}             % 枠線など
\usepackage{here}               % 画像の強制配置

% === 設定 ===
\title{\textbf{深層学習の数理的・技術的基盤の構築}\\ \large 〜ブラックボックスからの脱却〜}
\author{総合政策学部 2026年度入学者:鈴木 真理}
\date{2026年3月 吉日}

\begin{document}

\maketitle

% === 概要 (Abstract) ===
\begin{abstract}
% TODO: 全体の要約を書く(最後に書くのが定石だが、指針として仮置きしてもよい)
本課題において、筆者は医療用AI「Polaris」の実装に向け、深層学習(Deep Learning)のブラックボックス性を解消するための数理的・技術的基盤の構築を行った。既存のAPIやライブラリに依存せず、PythonとNumPyのみを用いてニューラルネットワークをゼロから実装することで、その内部動作を完全に掌握することを目的とした。本稿では、1月から2月にかけての学習成果、特に誤差逆伝播法の数理的導出と実装プロセスについて詳述し、3月に向けた自然言語処理(NLP)への応用計画を示す。
\end{abstract}

% === 本論 ===

\section{序論:なぜ「車輪の再発明」が必要か}

\subsection{背景:医療AI「Polaris」の課題}
% TODO: AO入試で提案したPolarisについて。API依存の現状と、医療現場で求められる「安全性」「説明責任」のギャップについて記述せよ。
% キーワード: Black Box, Accountability, Medical AI


\subsection{目的:ブラックボックスからの脱却}
% TODO: 本課題のゴール設定。「動けばいい」ではなく「内部構造を数学的に説明できる」状態を目指す旨を宣言せよ。


\section{数理的基盤の習得(1月の成果)}

\subsection{線形代数と計算グラフの接続}
% TODO: 行列演算(Dot Product)が、NNにおいてどのような意味(空間変換)を持つか。
% TODO: パーセプトロンの限界と、多層化・非線形化(活性化関数)の意味について。


\subsection{微分積分と最適化の視覚化}
% TODO: 勾配降下法の仕組み。「山を下る」直感と、数式(W = W - lr * dW)のリンク。
% TODO: なぜ数値微分ではなく、解析的な微分(誤差逆伝播)が必要なのか?


\section{ゼロからの実装と技術的獲得(2月の成果)}
\label{sec:implementation}

\subsection{NumPyによる完全実装}
% TODO: PyTorch禁止縛りの意義。
% TODO: 技術的詳細(ブロードキャスト、ベクトル化による高速化など)を具体的に。


\subsection{誤差逆伝播法(Backpropagation)の導出}
% TODO: 連鎖律(Chain Rule)の理解。
% TODO: 以下の数式が何を意味しているか、自分の言葉で解説せよ。
計算グラフを用いた連鎖律(Chain Rule)の理解により、複雑な微分の導出が可能となった。例えば、Affineレイヤの逆伝播は以下のように定式化される。

\begin{equation}
    \frac{\partial L}{\partial \bm{X}} = \frac{\partial L}{\partial \bm{Y}} \cdot \bm{W}^{\mathrm{T}}
\end{equation}

% TODO: ここに解説を書く(行列の形状チェック、転置の意味など)。


\section{成果と考察}

\subsection{MNISTデータセットによる検証}
% TODO: 定量的な成果(精度XX%)。
% TODO: 学習曲線(Lossの推移)についての考察(過学習は起きたか?など)。


\subsection{ブラックボックスは解消されたか}
% TODO: 定性的な成果。マインドセットの変化。「魔法」が「数理」に変わった実感。


\section{今後の展望:3月の計画}

\subsection{CNNからRNN/LSTMへ}
% TODO: 画像(空間的なパターン)から時系列(時間的なパターン)への拡張計画。


\subsection{自然言語処理とTransformerへの挑戦}
% TODO: Polaris(対話型AI)への接続。Attention機構の理解と、LLMアーキテクチャ設計への意欲。


% === 参考文献 ===
\begin{thebibliography}{99}
    \bibitem{saito1} 斎藤康毅『ゼロから作るDeep Learning ―Pythonで学ぶディープラーニングの理論と実装』オライリー・ジャパン, 2016.
    \bibitem{goodfellow} Ian Goodfellow et al., \textit{Deep Learning}, MIT Press, 2016.
    \bibitem{saito2} 斎藤康毅『ゼロから作るDeep Learning ② ―自然言語処理編』オライリー・ジャパン, 2018.
\end{thebibliography}

\end{document}